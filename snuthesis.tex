%%%%%%%%%%%%%%%%%%%%%%%%%%%%%%%%%%%%%%%%%%%%%%%%%%%%%%%%%%%%%%%%%%%%%%%%%%%%%%%%
%% 서울대학교 데이터마이닝연구실 구성원들의 박사학위논문 작성을 위해 아래 저작자의 자료를 일부 수정하였습니다.
%% Author: zeta709 (zeta709@gmail.com) 
%%%%%%%%%%%%%%%%%%%%%%%%%%%%%%%%%%%%%%%%%%%%%%%%%%%%%%%%%%%%%%%%%%%%%%%%%%%%%%%%


\RequirePackage{fix-cm} 
% oneside : 단면 인쇄용
% twoside : 양면 인쇄용
% phd : 박사
% openright : 챕터가 홀수쪽에서 시작
\documentclass[oneside,phd]{snuthesis_utf8}

%%%%%%%%%%%%%%%%%%%%%%%%%%%%%%%%%%%%%%%%
%% 목차 양식을 변경하는 코드
%% subfigure (subfig) package 사용 여부에 따라
%% tocloft의 옵션을 다르게 지정해야 한다.
%\usepackage[titles,subfigure]{tocloft} % when you use subfigure package
\usepackage[titles]{tocloft} % when you don't use subfigure package
\makeatletter % don't delete me
\renewcommand\cftchappresnum{Chapter~}
\renewcommand\cftfigpresnum{Figure~}
\renewcommand\cfttabpresnum{Table~}


\usepackage[pdftex,bookmarks=true]{hyperref}

\makeatother % don't delete me
\newlength{\mytmplen}
\settowidth{\mytmplen}{\bfseries\cftchappresnum\cftchapaftersnum}
\addtolength{\cftchapnumwidth}{\mytmplen}
\settowidth{\mytmplen}{\bfseries\cftfigpresnum\cftfigaftersnum}
\addtolength{\cftfignumwidth}{\mytmplen}
\settowidth{\mytmplen}{\bfseries\cfttabpresnum\cfttabaftersnum}
\addtolength{\cfttabnumwidth}{\mytmplen}
%% 목차 양식을 변경하는 코드 끝
%%%%%%%%%%%%%%%%%%%%%%%%%%%%%%%%%%%%%%%%

%%%%%%%%%%%%%%%%%%%%%%%%%%%%%%%%%%%%%%%%
%% 다른 패키지 로드
%% http://faq.ktug.or.kr/faq/pdflatex%B0%FAlatex%B5%BF%BD%C3%BB%E7%BF%EB
%% 필요에 따라 직접 수정 필요
\ifpdf
	% \input glyphtounicode\pdfgentounicode=1 %type 1 font사용시
	%\usepackage[pdftex,unicode]{hyperref} % delete me
	%\usepackage[pdftex]{graphicx}
	%\usepackage[pdftex,svgnames]{xcolor}
\else
	%\usepackage[dvipdfmx,unicode]{hyperref} % delete me%
	%\usepackage[dvipdfmx]{graphicx}
	%\usepackage[dvipdfmx,svgnames]{xcolor}
\fi
%%%%%%%%%%%%%%%%%%%%%%%%%%%%%%%%%%%%%%%%
%
%% \title : 22pt로 나오는 큰 제목
%% \title* : 16pt로 나오는 작은 제목
\title{Dissertation Title Dissertation Title \\abcdefghijk}
\title*{박사학위논문 한글제목}

\titlen{Dissertation Title Dissertation Title abcdefghijk}

\author{Gildong~Hong}
\author*{홍길동} % Same as \author.
\authorn{홍~길~동}
\phonenumber{010-0000-0000}
\studentnumber{2000-00000}
\advisor{Nada~Ga}
\advisor*{가나다}
\advisorn{가~나~다}
\graddate{2015~년~~8~월}
\submissiondate{2015~년~~6~월}
\submissiondaten{2015~년~~8~월~~10~일}
\approvaldate{2015~년~~6~월}

\committeemembers%
{교 수 님}%
{교 수 님}%
{교 수 님}%
{교 수 님}%
{교 수 님} %

%% Length of underline
\setlength{\committeenameunderlinelength}{5cm}

\begin{document}
\pagenumbering{Roman}
\makefrontcover
\makeapproval

%agreement page
\cleardoublepage
\makeagreement

\cleardoublepage
\pagenumbering{roman}

\keyword{SNU, dmlab, thesis}
\begin{abstract}
In this dissertation, ...

\end{abstract}


% % 여기 수정할 것
\tableofcontents
\cleardoublepage
\addcontentsline{toc}{chapter}{\contentsname}

\listoftables
\cleardoublepage
\addcontentsline{toc}{chapter}{\listtablename}

\listoffigures
\cleardoublepage
\addcontentsline{toc}{chapter}{\listfigurename}


\cleardoublepage
\pagenumbering{arabic}

\chapter{Introduction}

introduction...

\section{Section A}
\section{Section B}


\chapter{Literature Review}
Main contents.
This book \cite{bishop2006} is ...

\section{Classification Algorithms}

\subsubsection{LDA}
\subsubsection{DT}

\begin{table}[!hbp]
\caption{table title}
\end{table}

\chapter{Chapter}

contents

\begin{figure}[!hbp]
\caption{figure title}
\end{figure}

\subsection{subsection title}
subsection



\chapter{Conculsion}
Conculsion.


\begin{bibpage}
	\bibliographystyle{ieeetr}
	\bibliography{ref}
\end{bibpage}



\keywordalt{서울대학교, 데이터마이닝연구실, 박사학위논문}
\begin{abstractalt}
한글 요약 내용이 여기에 들어갑니다.
\end{abstractalt}

\acknowledgement
Thanks!

\end{document}

